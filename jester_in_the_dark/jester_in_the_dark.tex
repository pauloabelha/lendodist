\documentclass[12pt]{article}
 \usepackage[margin=1in]{geometry} 
\usepackage{amsmath,amsthm,amssymb,amsfonts}
\usepackage{mathtools,amssymb}
\usepackage{color}
\usepackage{multirow}
\usepackage{rotating}

\newcommand{\N}{\mathbb{N}}
\newcommand{\Z}{\mathbb{Z}}
 
\newenvironment{problem}[2][Problem]{\begin{trivlist}
\item[\hskip \labelsep {\bfseries #1}\hskip \labelsep {\bfseries #2.}]}{\end{trivlist}}
%If you want to title your bold things something different just make another thing exactly like this but replace "problem" with the name of the thing you want, like theorem or lemma or whatever

\begin{document}
 
%\renewcommand{\qedsymbol}{\filledbox}
%Good resources for looking up how to do stuff:
%Binary operators: http://www.access2science.com/latex/Binary.html
%General help: http://en.wikibooks.org/wiki/LaTeX/Mathematics
%Or just google stuff

\definecolor{blogger-black}{rgb}{0.1025,0.1025,0.1025}
\pagecolor{blogger-black}
\color{white} 
 
\title{Jester in the Dark}
\author{Paulo Abelha}
\maketitle

$g_1$ : First drawn coin is gold

$g_2$ : Second drawn coin is gold

$c_1$ : Chest with 2 gold coins is open

$c_2$ : Chest with 1 gold and 1 silver coin is open

$c_3$ : Chest with 2 silver coins is open

\begin{tabular}{|c|c|c|c|}
\cline{2-4}
\multicolumn{1}{c|}{} & $c_1$ & $c_2$ & $c_3$ \\
\hline
$g_1$ & $\frac{2}{6}$ & $\frac{1}{6}$ & $\frac{0}{6}$ \\
\hline
$\bar{g_1}$ & $\frac{0}{6}$ & $\frac{1}{6}$ & $\frac{2}{6}$  \\
\hline
\end{tabular}
\begin{tabular}{|c|c|c|c|}
\cline{2-4}
\multicolumn{1}{c|}{} & $c_1$ & $c_2$ & $c_3$ \\
\hline
$g_2$ & $\frac{2}{6}$ & $\frac{1}{6}$ & $\frac{0}{6}$ \\
\hline
$\bar{g_2}$ & $\frac{0}{6}$ & $\frac{1}{6}$ & $\frac{2}{6}$  \\
\hline
\end{tabular}

\begin{gather}
P(g_2|g_1) = \sum_{i=1}^{3} P(g_2,c_i|g_1)
\end{gather}

\begin{gather}
P(g_2|g_1) = \sum_{i=1}^{3} P(g_2|c_i,g_1) P(c_i|g_1)
\end{gather}

\begin{gather}
P(g_2|c_1,g_1) = 1
\end{gather}

\begin{gather}
P(g_2|c_2,g_1) = 0
\end{gather}

\begin{gather}
P(g_2|c_3,g_1) = 0
\end{gather}

\begin{gather}
P(c_i|g_1) = \frac{P(g_1|c_i) P(c_i)}{P(g_1)}
\end{gather}

\begin{gather}
P(c_i) = \frac{1}{3}
\end{gather}

\begin{gather}
P(g_1|c_1) = 1
\end{gather}

\begin{gather}
P(g_1|c_2) = \frac{1}{2}
\end{gather}

\begin{gather}
P(g_1|c_3) = 0
\end{gather}

\begin{gather}
P(g_1) = \sum_{i=1}^{3} P(g_1|c_i)P(c_i)
\end{gather}

\begin{gather}
P(g_1) = \frac{1}{3}\Big(1 + \frac{1}{2} + 0\Big) = \frac{1}{2}
\end{gather}

\begin{gather}
P(c_1|g_1) = \frac{P(g_1|c_1) P(c_1)}{P(g_1)} = \frac{2}{3}
\end{gather}

\begin{gather}
P(c_2|g_1) = \frac{1}{3}
\end{gather}

\begin{gather}
P(c_3|g_1) = 0
\end{gather}

\begin{gather}
P(g_2|g_1) = (1 * \frac{2}{3}) + (0 * \frac{1}{3}) + (0 * 0) = \frac{2}{3}
\end{gather}

\end{document}
